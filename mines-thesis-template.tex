% Please compile the tex file by pdfLaTeX, 
% it would handle the landscape pages correctly.
\documentclass[12pt]{mines-thesis}
% Font Size: 10-12 point type, change the value above

%========================================
%                              Set-up                              
%========================================
% Text will be double spaced or 1.5 spaced. 
\OnehalfSpacing 
%\DoubleSpacing
% Set the indentation of each paragraph.
\setlength{\parindent}{2em}
% Set the lowest level to show in the table of contents
\maxtocdepth{subsubsection}
% Set the fully-justified or left-justified text
%\textalignment{fully-justified}
%\textalignment{left-justified}

%========================================
%                           Packages                             
%========================================
\usepackage{lipsum} % dummy text
\usepackage{graphicx} % for images
% hyperref is a package that frequently requires compiling twice 
\usepackage{hyperref} % for \url and links
\urlstyle{same} % use same font of url
\hypersetup{hidelinks} % to remove the red border for hyperlinks
\usepackage{pdflscape} % for landscape mode page
\usepackage{everypage} % for landscape mode page numbering 
\usepackage{longtable} % for really long tables

% Place all chapter tex files in ``Chapters'' folder
% Place all references bib files in ``References'' folder

% Place all figures in ``Figures'' folder
\graphicspath{{Figures/}} 
% Or define several directories to be searched for figures.
% \graphicspath{{Figures/}{../figures/}{C:/Users/me/Documents/project/figures/}}

\usepackage{enumitem}
\newlist{abbrv}{itemize}{1}
\setlist[abbrv,1]{label=,labelwidth=1in,align=parleft,itemsep=0.1\baselineskip,leftmargin=!}

\begin{document}
%========================================
%                            Front Matter                              
%========================================
\autotitle %automatically change the title into upper cases and reshape it to an inverted pyramid
%\title{Thesis title centered on the page vertically and horizontally, in all upper case letters and in an inverted pyramid shape. Math mode: $\frac{2^{15}}{\pi}$}
% If you want to break the title by yourself, you must use ``\protect\\ ''.
% For example,
\title{
	Thesis title centered on the page vertically and horizontally, \protect\\
	in all upper case letters and in an inverted\protect\\
	pyramid shape. 	Math mode: $\frac{2^{15}}{\pi}$
}
	
\author{Student Name}  % this is your name exactly as you want it
\year{2020}     % this is the year of defence
\degree{Doctor of Philosophy}{Computer Science} %
\advisor{Dr. Thesis Advisor}
\coadvisor{Dr. Co-Advisor}  % comment out this line  if no co-advisor
\department{Department of XXX}
\departmenthead{Dr. Department Head}
	
	
%==================Abstracts===============
% 1. Are generally 200-300 words in length, 
% 2. Consist of one to two paragraphs of information,
% 3. Does not usually contain citations, 
% 4. Do not repeat the thesis title, and
% 5. Each paragraph should be indented.	
\begin{abstract}
	The abstract is a concise, one to three sentence statement of the thesis problem, a brief description consisting of no more than a few sentences describing the research method or design, and a report of the major findings and conclusions.
	%
			
	%
	The abstract submitted online at the time of thesis submission should be the same as the abstract inside the thesis. ProQuest continues to publish print indexes that have maximum abstract lengths of 150 words for Masters and 350 words for PhDs. Abstracts exceeding these limits will be truncated in the print indexes so it may be wise to work within those word limits in most cases.
\end{abstract}

%%============Acknowledgements==============
% Comment the section out if you don't need it 
%\begin{acknowledgment}
%	This optional page includes a paragraph or two acknowledging and thanking your
%	advisor(s), committee members, funding sponsors, family members etc.
%			
%	Typically, if you acknowledge everyone here, you won’t have a dedication page.
%	
%	Below are dummy texts generated  by the \texttt{lipsum} package.
%	\lipsum[3]
%\end{acknowledgment}

%%==================Dedication===============
% Comment the section out if you don't need it 
\begin{dedication}
	Dedication Page: A dedication page is optional and not frequently included in a thesis. However, occasionally the thesis writer wants to dedicate the document to a professional colleague, friend, or relative. A dedication typically expresses gratitude for someone's support. If a dedication page is included, it is placed at the end of the front matter section, following the acknowledgments. Typically, a dedication page has no title, it simply states, e.g., "For my father." Roman numeral page numbering continues on the dedication page.
\end{dedication}	


\renewcommand{\makecft}{	
	\tableofcontents*
	\clearpage	
	\listoffigures
	\clearpage
	\listoftables
	\clearpage
	\listofsymbols
	\clearpage
	\clearpage
}

\newcommand{\listofsymbols}{
\chapter*{List of Abbreviations}
\chaptermark{List of Abbreviations}

\begin{abbrv}
	\item[ADL]                   Activities of Daily Life
	\item[AST]                   Alternate Step Test
	\item[BMI]                   Body Mass Index
	\item[CSFT]                  Cross Step moving on Four Stops
	\item[DBN]                   Dynamic Bayesian Networks
	\item[DFRAC]                 Demura's Fall Risk Assessment Chart
	\item[EMG]                   Electromyography
	\item[FEUP]                  Faculdade de Engenharia da Universidade do Porto
	\item[FPRI]                  Fall Prediction and Risk Index
	\item[FR]                    Fall Probability
	\item[FRI]                   Fall Risk Index
	\item[GDP]                   Gross Domestic Product
	\item[GUGT]                  Get-Up-ang-Go Test
	\item[LABIOMEP]              Laboratório de Biomecânica do Porto
	\item[MEMs]                  Micro-Electromechanics
	\item[MTC]                   Minimum Toe Clearance
	\item[PCA]                   Principal Components Analysis
	\item[PPA]                   Physiological Profile Assessment
	\item[PPP]                   Purchasing Power Parities
	\item[SMWT]                  Six Meter Walking Test
	\item[STRATIFY]              Saint Thomas's Risk Assessment Tool in Falling Elderly Inpatients
	\item[STST]                  Sit-To-Stand Test
	\item[STST5]                 Sit-To-Stand Test with 5 repetitions
	\item[SVM]                   State Vector Machine
	\item[SWHSA]                 Smart Wearable Health Systems and Applications
	\item[TUGT]                  Timed Up-and-Go Test
	\item[USB]                   Universal Serial Bus
	\item[USUST]                 Unstructured and Unsupervised Test
	\item[WEFAPS]                Wearable Fall Assessment \& Prediction System
	\item[WHO]                   World Health Organization
\end{abbrv}
}

\makefrontmatter


%========================================
%                              Main Body                              
%========================================
%TODO Main Body
	
% You can separate the chapters to different files

%========================================
%                            Chapter                            
%======================================== 
\chapter{Introduction}
\lipsum[2]

%========================================
%                            Section                             
%======================================== 	 
\section{First section}
\lipsum[3]

%========================================
%                            Section                             
%======================================== 
\section{Sencond  section}
\lipsum[3]
\subsection{First SubSection}
\lipsum[3]
\subsection{Second SubSection}
\lipsum[3]
\subsubsection{Sub SubSection}
\lipsum[3]	 
	 
%========================================
%                            Section                             
%======================================== 
\section{Last section Math mode: $\frac{2^{15}}{\pi}$} 
\lipsum[3]
\section{Last section to test Citation} 

First paper~\cite{goossens1994latex} did something. Note that here we use \verb|~| to make sure the citation not go to the other line. 

Lamport et al.~\cite{lamport1994latex} did something.

Many papers~\cite{makuuchi2000progress,yassin1994latex} did something.


All papers~\cite{makuuchi2000progress,yassin1994latex,
	goossens1994latex,lamport1994latex} did something.

Online link~\cite{onlineWindows} is about something.

More papers: \cite{colu92}, \cite{goossens1994latex}, \cite{jame76}, \cite{colu92,phil99,gree00,smit54}

%========================================
%                            Chapter                            
%======================================== 
\chapter{Reproduced Chapter:  This chapter represents that style used when a journal article has already been published or has been accepted for publication }
\reproduceinfo{
	Reproduced with permission from The Journal of Geology\\
	2012 Elsevier Ltd. 
	Katherine Smith\symbolfootnote{1}{Primary author and editor.}\footnote[1]{Department of  Civil and Environmental Engineering, Colorado School of Mines, 1500 Illinois Street, Golden, Colorado 80401, USA.}, 
	Eric Wright\symbolfootnote{2}{Corresponding author. Direct correspondence to \texttt{example@mines.edu}.}\footnote[2]{Southern Nevada, 550 City Parkway, Suite 810, Las Vegas, NV, 89106, USA.}
}
%\reproduceabstract % Print Abstract at the left




\section{Abstract}
This is the original abstract of your paper.This is the original abstract of your paper. This is the original abstract of your paper.This is the original abstract of your paper.

\lipsum[4]

%========================================
%                            Section                             
%======================================== 
\section{First section} 
\lipsum[2]

%========================================
%                            Chapter                            
%======================================== 
\chapter{This is a really really long title that someone else wrote for all the penguins in the world Math mode: $\frac{2^{15}}{\pi}$}
\lipsum[2]

%========================================
%                            Section                             
%======================================== 	
\section{First section }
\lipsum[3]

%========================================
%                            Section                             
%======================================== 
\section{This is a really really really really really really really really really really really really really really really really really really really really really really really really really really really really really really long title having more than three lines of text to appear on the toc.}
\lipsum[3-4]

%========================================
%                            Section                             
%======================================== 	 
\section{Last section} 
\lipsum[4]

\chapter{Example of Long  Long Long Long Long Long Long Long Long Long Long Long Long Long Long Long Long Long Long Long Long Long Long Long Long Long chapter}
\lipsum[4]
	
\section{First section}
\lipsum[2]
\subsection{Subsection}
\lipsum[5]
\subsection{This is a really really really really really really really really really really really really really really really really really really really really really really really really really really really really really really long title having more than three lines of text to appear on the toc.}
\lipsum[4]
\subsubsection{SubSubsection}
\lipsum[5]

\section{Second section}
\lipsum[4]

\section{Last section} 
\lipsum[6]



%========================================
%                            Chapter                            
%======================================== 
\chapter{Figure and Table Chapter}
\lipsum[5]
%========================================
%                            Section                             
%======================================== 
\section{First section}
\lipsum[2-3]

%========================================
%                            Section                             
%======================================== 
\section{Second section} 
\lipsum[4-5]

%========================================
%                            Section                             
%======================================== 
\section{Last section}
\lipsum[6-7]
%========================================
%                            Chapter                            
%======================================== 
\chapter{Last chapter}
\lipsum[1]
%========================================
%                            Section                             
%======================================== 
\section{First section} 
\lipsum[2-3]


%========================================
%                            Section                             
%======================================== 
\section{Second section} 
\lipsum[4-5]

%========================================
%                            Section                             
%======================================== 
\section{Last section to test Citation} 

First paper~\cite{goossens1994latex} did something. Note that here we use \verb|~| to make sure the citation not go to the other line. 

Lamport et al.~\cite{lamport1994latex} did something.

Many papers~\cite{makuuchi2000progress,yassin1994latex} did something.


All papers~\cite{makuuchi2000progress,yassin1994latex,
	goossens1994latex,lamport1994latex} did something.

Online link~\cite{onlineWindows} is about something.


% Or add chapters here
\chapter{Last Last chapter}
\lipsum[3]
\section{Section}
\lipsum[4]
\subsection{SubSection} 
\lipsum[4]
\subsubsection{SubSubSection}
\lipsum[4-5]

\linespace
	
%========================================
%                             Back Matter                               
%========================================
%%================Reference===============
% You may:
% 1. Add one reference section at the end of the thesis/dissertation, 
% or 
% 2. Add a reference section at the end of each chapter
% Set the NAME of your bibliography file
%\bibliography{References/referenceAll.bib} 
% Set the STYLE of your reference, you need to compile TWICE
% entries are sorted alphabetically and labelled with numbers.
%\bibliographystyle{plain}
\begin{thebibliography}{10}
	
	\bibitem{colu92}
	Christopher Columbus.
	\newblock {\em How {I} Discovered {America}}.
	\newblock Hispanic Press, Barcelona, 1492.
	
	\bibitem{goossens1994latex}
	Michel Goossens, Frank Mittelbach, and Alexander Samarin.
	\newblock {\em The LATEX companion}.
	\newblock Addison-Wesley Reading, 1994.
	
	\bibitem{gree00}
	R.~J. Green, U.~P. Fred, and W.~P. Norbert.
	\newblock Things that go bump in the night.
	\newblock {\em Psych. Today}, 46:345--678, 1900.
	
	\bibitem{jame76}
	Kelly James, George Harris, Jr., and Wilby Wollops.
	\newblock {American} independence and magnetism.
	\newblock {\em Revol. Tracts}, 32:34--55, 1776.
	
	\bibitem{lamport1994latex}
	Leslie Lamport.
	\newblock {\em LATEX: a document preparation system: user's guide and reference
		manual}.
	\newblock Addison-wesley, 1994.
	
	\bibitem{onlineWindows}
	MultiMedia LLC.
	\newblock {MS Windows NT} kernel description.
	\newblock
	\url{http://web.archive.org/web/20080207010024/http://www.808multimedia.com/winnt/kernel.htm},
	1999.
	\newblock Accessed: 2010-09-30.
	
	\bibitem{makuuchi2000progress}
	Keizo Makuuchi.
	\newblock Progress in radiation vulcanization of natural rubber latex.
	\newblock Technical report, 2000.
	
	\bibitem{phil99}
	T.~P. Phillips.
	\newblock Possible influence of the magnetosphere on {American} history.
	\newblock {\em J. Oddball Res.}, 98:1000--1003, 1999.
	
	\bibitem{smit54}
	J.~G. Smith and H.~K. Weston.
	\newblock Nothing particular in this year's history.
	\newblock {\em J. Geophys. Res.}, 2:14--15, 1954.
	
	\bibitem{yassin1994latex}
	Mohamed~S Yassin, Michelle~B Lierl, Thomas~J Fischer, Kevin O'Brien, Jennifer
	Cross, and Christine Steinmetz.
	\newblock Latex allergy in hospital employees.
	\newblock {\em Annals of allergy}, 72(3):245--249, 1994.
	
\end{thebibliography}


%%=================Appendix===============
% The \appendix declaration changes the numbering of chapters 
% to an alphabetic form and also changes the names of chapters 
% from \chaptername (default  CHAPTER) to the value 
% of \appendixname (default APPENDIX). 
\appendix 
\chapter{Supplement Data}\label{appendix:1}

Appendix material is information that is not
essential to the text but that contributes to it.

Appendices are used to include information such as the following:

\begin{itemize}
	\item Original data
	\item Long quotations 
	\item Supporting legal decisions or laws 
	\item Computer codes and programs 
	\item Lithologic and petrographic descriptions
	\item Questionnaires 	
	\item Forms and documents 
	\item Permissions to use copyrighted material
	\item Long tables
\end{itemize}


All Figuires and  Tables in an Appendix need to be labeled with a number \& a caption and need to be listed in either the List of Figures or List of Tables.

You may include long appendices as electronic attachments to your thesis. In either instance, appendices are listed in the table of contents.

\begin{figure}[h]
	\centering
	\includegraphics[width=\linewidth]{fig4}
	\caption{Example Figure in Appendix~\ref{appendix:1}}
\end{figure}


\lipsum[5-9]
\chapter{Supplemental Electronic Files}\label{appendix:2}
 The Appendix for the Supplemental Electronic Files should look like the example below. Be sure to include the Appendix in the Table of Contents.
 
 \begin{table}[h]
 	\caption{Example Table in Appendix~\ref{appendix:2}}
 	\label{tab:appendixtable}
 	\centering
 	\begin{tabular}{ccl}
 		\toprule
 		Non-English or Math & Frequency   & Comments          \\
 		\midrule
 		\O                  & 1 in 1,000  & For Swedish names \\
 		$\pi$               & 1 in 5      & Common in math    \\
 		\$                  & 4 in 5      & Used in business  \\
 		$\Psi^2_1$          & 1 in 40,000 & Unexplained usage \\
 		\bottomrule
 	\end{tabular}
 \end{table}

\end{document}